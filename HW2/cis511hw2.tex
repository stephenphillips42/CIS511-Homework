\documentclass[english]{article}

\usepackage[latin9]{inputenc}
\usepackage[letterpaper]{geometry}
\geometry{verbose,tmargin=1in,bmargin=1in,lmargin=1in,rmargin=1in}
\usepackage{amsmath}
\usepackage{amssymb}
\usepackage{tikz}
\usetikzlibrary{automata,positioning}

\tikzstyle{dir}=[->, very thick]
\tikzstyle{circ}=[draw, circle, very thick]

\title{CIS 511 Homework 2}
\author{Stephen Phillips, Dagaen Golomb}
\date{\today }


\begin{document}
\maketitle
\subsection*{Problem 1}
\subsubsection*{Part a}
First we show that $A = \{ \mathbf{a}^m \mathbf{b}^n \mathbf{c}^n | m,n \ge 0 \}$ is context free. The grammar is as follows:
\begin{align*}
S \rightarrow & XY \\
X \rightarrow & \varepsilon \; | \; aX \\
Y \rightarrow & \varepsilon \; | \; bYc \\
\end{align*}
Clearly this generates only strings in the language, and every string in $A$ can be mapped to this by the number of generations of $X$ and $Y$ corresponding to $m$ and $n$ respectively. The language $B = \{ \mathbf{a}^n \mathbf{b}^n \mathbf{c}^m | m,n \ge 0 \}$ is almost identical:
\begin{align*}
S \rightarrow & XY \\
X \rightarrow & \varepsilon \; | \; aXb \\
Y \rightarrow & \varepsilon \; | \; cY \\
\end{align*}
With the proof that for this being the same as $A$. So these two languages are context free, however their intersection $C = A \cap B = \{ \mathbf{a}^n \mathbf{b}^n \mathbf{c}^n | n \ge 0 \}$ is not context free, as we showed in class.

\subsubsection*{Part b}
We show that the complement of a context free language is not necessarily context free. Suppose not, i.e. that CFLs are closed under complementation. Now, for any context free languages $A$ and $B$, $A \cup B$ would be context free, since context free languages are closed under union. Then $\bar{A \cup B} = \bar{A} \cap \bar{B}$ would also be in the language. Since we assume that context free languages are closed under complement $\bar{A}$ and $\bar{B}$ would be context free. But since $A$ and $B$ are arbitrary that means the intersection of any two context free languages is context free, a contradiction (proved false in part A above). Therefore context free languages are not closed under complement.

\section*{Problem 2}
The context free grammar $G$ of $L = \{ w\#x | w^\mathcal{R} \textrm{is a substring of }x, x \in \{0,1\}^* \}$ is:
\begin{align*}
S \rightarrow & aSXaX \; | \; bSXbX \; | \; \# \\
X \rightarrow & \varepsilon \; | \; aX \; | \; bX \\
\end{align*}
Now we need prove this is correct, or in other words $L(G) \subseteq L$ and $L \subseteq L(G)$.
\begin{itemize}
\item $L(G) \subseteq L$. This generates strings of the form
      $\sigma_1 \sigma_2 \ldots \sigma_n \# \{a,b\}^* \sigma_n \{a,b\}^* \sigma_{n-1} \ldots \{a,b\}^* \sigma_1 \{a,b\}^* = w\#x$.
      Each generation of $S$ creates some $\sigma_i$ on the left and right, with the right having the reverse order of the left.
      The variable $X$ can generate any string in $\{a,b\}^*$, which generates the strings between the characters on the left side.
      So by construction $w^\mathcal{R}$ is a substring of $x$ as every $\sigma_i$ in $w$ is in $x$ and in the same order. Therefore
      the language this grammar generates is a subset of $L$
\item $L \subseteq L(G)$. Again the grammar generates strings of the form 
      $\sigma_1 \sigma_2 \ldots \sigma_n \# \{a,b\}^* \sigma_n \{a,b\}^* \sigma_{n-1} \ldots \{a,b\}^* \sigma_1 \{a,b\}^* = w\#x$.
      By definition if $x$ is a substring of $y$ then there exists $z_1, z_2, \ldots, z_n, z_{n+1} \in \{a,b\}^*$ such that 
      $y = z_1 x_1 \ldots z_n x_n z_{n+1}$. So for every string $s = w\#x \in L$ map $w_1$ to $\sigma_n$, and in general $w_i$ to
      $\sigma_{n-i+1}$ and then the $z_i$ map to the $\{a,b\}^*$. 
\end{itemize}

% I am not sure how good this proof is, Dagaen. If you can make it better please do so!

\subsection*{Problem 3}
Show that adding the rule \(S \rightarrow SS\) fails to show that context free languages are closed under Kleene star.

The easiest way to show this is to show that this production does not exhibit the behavior of the Kleene star. This fails because it is possible that the original grammar did not have $S$ able to go to a start symbol. For instance
the language \(L = \{ a^n b \mid n > 0 \}\) can be generated by the following grammar:
\[ S \rightarrow \; aS \; | \; ab \]
Adding the rule $SS$ makes this:
\[ S \rightarrow \; aS \; | \; ab \; | \; SS \]
Which is the language \(L' = \{ x^m \mid x \in L, m > 0 \}\) whereas \(L^* = L' \cup \{\varepsilon\}\).

Another easy example is any CFL that does not include $\varepsilon$, adding this production will produce another grammar that does not contain $\varepsilon$. But the Kleene star can always generate $\varepsilon$ by definition, so this production does not exhibit the behavior of the Kleene star. Therefore, it cannot be used to prove closure under such.

\subsection*{Problem 4}
Changing the grammar to new add new start symbol $S'$ and add to the production rules:
\[ S' \rightarrow \; SS' \; | \; \varepsilon \]

This makes an arbitrary number of strings in the language concatenated, including 0.

\subsection*{Problem 5}
Find a context free grammar for the language \(L = \{ x | \textrm{\#a's is 2 times \#b's}\}\)
\begin{align*}
S \rightarrow & aSbSa \;|\; bSaSa \;|\; aSaSb \;|\; SS \;|\; \varepsilon
\end{align*}

Proof by induction: Note strings of length 0 ($\varepsilon$) hold, since 0 = 2*0. The next valid strings are of length 3 and include only $aab, aba,$ and $baa$. So this holds for the smallest 4 possible strings. Now assume productions up to n have generated strings in L. For production n+1, we either add 0 a's and 0'b, hence keeping twice as many a's as b's, or we add 2 a's and 1 b. In the latter case, we still maintain that we have twice as many a's as b's. Therefore, this production only produces strings with twice as many a's as b's. Note that since we include all permutations of the a's and b's and allow insertion between any two (including possibly with the empty string), this generates all such strings.

\subsection*{Problem 6}
Show that \(L = \{ xy \mid x,y \in \{0,1\}^* |x| = |y|, x \neq y \}\) is context free

The following grammar generates the language
\begin{align*}
S \rightarrow & S_0 S_1 | S_1 S_0 | \varepsilon \\
S_0 \rightarrow & B S_0 B | 0 \\
S_1 \rightarrow & B S_1 B | 1 \\
B \rightarrow & 0 | 1
\end{align*}

Proof:
\begin{itemize}
\item \(L(G) \subseteq L\). Strings generated by this language are of the form
 \[\{0,1\}^n 1 \{0,1\}^n \{0,1\}^m 0 \{0,1\}^m = \{0,1\}^n 1 \{0,1\}^m \{0,1\}^n 0 \{0,1\}^m\] 
 (or switch the 1 and the 0, it is equivalent). We can see that \(x = \{0,1\}^n 1 \{0,1\}^m\) and
 \(y = \{0,1\}^n 0 \{0,1\}^m\). as the $(n+1)^{th}$ bit differs, these strings cannot be the same.
  However they are both of length $m+n+1$. Therefore this string $xy$ is in the language.
\item \(L \subseteq L(G)\). Suppose we are given two strings $x, y$ such that $|x| = |y|, x \neq y$. Then by
 definition there must exist one bit where they differ, say the $(i+1)^{th}$ bit, with the strings being of
 length $i+j+1$. Then the string $xy$ is of the form \( \{0,1\}^i b \{0,1\}^j \{0,1\}^i \bar{b} \{0,1\}^j \),
 where $\bar{b}$ is the complement of $b$. Then we can generate with the above grammar using $i$ generations
 of $S_b$ and $j$ generations of  $S_{\bar{b}}$ with the appropriate ordering, since 
 \( \{0,1\}^i b \{0,1\}^j \{0,1\}^i \bar{b} \{0,1\}^j = \{0,1\}^i b \{0,1\}^i \{0,1\}^j \bar{b} \{0,1\}^j \).
 Therefore we can generate any string in this language.
\end{itemize}

\subsection*{Problem 7}
% Implements the above grammar, so pretty straightforward. Needs an S_0 and S_1 state basically
To implement the above grammar we use the machine \(M = (Q,\Sigma,\Gamma,q_0,\delta,F)\) implemented in
the same fasion of converting a grammar to a machine shown in class. The machine is as follows:

\begin{align*}
\Sigma &=\; \{0,1\} \\
\Gamma &=\; \{0,1,\$,S_0,S_1,B\} \\
   Q   &=\; \{q_{start},q_{end},q_{c},q_{S_0},q_{S_1},q_{S_0B},q_{S_1B}\} \\
  q_0  &=\; q_{start} \\
   F   &=\; \{q_{end}\}
\end{align*}
Each of the states has a meaning. The states $q_{start}$ and $q_{end}$ have obvious meanings. The state
$q_c$ means the computation state, $q_{S_0}$ means the states where we add the rules for $S_0$. Similarly
for $q_{S_1}$. The states $q_{S_0B},q_{S_1B}$ are intermediate states to push more symbols on the stack.

And now for the transition function:
\begin{align*}
\delta(q_{start},\varepsilon,\varepsilon) = \{(q_{c},\$)\} & \textrm{Pushing the end of stack symbol} \\
\delta(q_{c},0,B) = \{(q_{c},\varepsilon)\} & \textrm{Implementing $B$'s rules} \\
\delta(q_{c},1,B) = \{(q_{c},\varepsilon)\}  \\
\delta(q_{c},\varepsilon,S_0) = \{(q_{S_0},B)\} & \textrm{Implementing $S_0$'s rules} \\
\delta(q_{S_0},\varepsilon,\varepsilon) = \{(q_{S_0B},S_0)\}  \\
\delta(q_{S_0B},\varepsilon,\varepsilon) = \{(q_{c},B)\}  \\
\delta(q_{c},\varepsilon,S_1) = \{(q_{S_1},B)\} & \textrm{Implementing $S_1$'s rules} \\
\delta(q_{S_1},\varepsilon,\varepsilon) = \{(q_{S_1B},S_1)\}  \\
\delta(q_{S_1B},\varepsilon,\varepsilon) = \{(q_{c},B)\}  \\
\delta(q_{c},\varepsilon,\$) = \{(q_{end},\varepsilon)\}  \\
\end{align*}

As this implements the above grammar, and we showed that this works in class, this implements the language
desired. I realized after writing this that I didn't have to go into all this detail, but since I finished
it already here it all is.

\subsection*{Problem 8}
Show that the language \(L = \{a^i b^j \mid i \neq j, 2i \neq j\}\) is context free.

This can be shown via the union of the CFLs that make up the 3 intervals of i and j who do not meet the constraint. Consider 3 cases:
\begin{enumerate}
\item \(i > j\)
\[ S \rightarrow aSb | aS^1 \]
\[ S^1 \rightarrow aS^1 | \varepsilon \]
This allows pairing i's and j's but must eventually follow $S^1$ to terminate which adds at least one extra a.

\item \(2i < j\)
\[ S \rightarrow aSbb | S^1 \]
\[ S^1 \rightarrow S^1b | \varepsilon \]
Similarly this allows pairing i's with 2 j's but must eventually follow $S^1$ to terminate which adds at least one extra b.

\item \(i < j < 2i\)
\[ S \rightarrow  aSb | aS^1b \]
\[ S^1 \rightarrow  aS^1bb | abb \]
This one takes some more thinking. It adds a's and b's equally, but then allows one to add some a's with 2 b's. Due to this, the closest one can get to equal is one less b than a's, or one can use many of the second production to get close to 1/2 from above (j/i) but not quite.

Since CFL's are closed under union, we can union these 3 CFL's to show that L is context free.

\end{enumerate}

\subsection*{Problem 9}
\subsubsection*{Part a}
Prove \(L = \{0^n 1^n 0^n 1^n \mid n \ge 0\}\) is not context free.

Proof: Suppose not. Let $p$ be the pumping lemma constant. Consider the string $s = 0^p 1^p 0^p 1^p$. By the pumping lemma
we have $s = uvxyz$ with $|vxy| \le p$ and $|vy| > 0$, and that \(\forall i \ge 0,\; uv^ixy^iz \in L\). By the first property
we know $vxy$ cannot contain more than two symbols. If you let $i=2$ you have for any possible $s' = uv^2xy^2z$ that we have
an uneven number of 0s and 1s in the section $v^2xy^2$ than in the sections it is not in. This means $s' \notin L$.
But by the pumping lemma $s' \in L$, a contradiction. Therefore $L$ is not context free.

\subsubsection*{Part d}
Prove \(L = \{t_1\#t_2\#\ldots\#t_k \mid k \ge 2, t_i \in \{0,1\}^* \textrm{ and } \exists i, j : i \neq j \land t_i = t_j \}\) is not context free.

% OK actually the example I came up with won't work because you can have an arbitrary number of #'s and 
% I don't see how you can rule out the possibility that v or y contain the # and repeat it with two
% strings that happen to be the same...
Proof: Suppose not. Let $p$ be the pumping lemma constant. Consider the string $s = 0^p1^p \# 0^p1^p$. By the pumping lemma we have $s = uvxyz$ with $|vxy| \le p$ and $|vy| > 0$, and that \(\forall i \ge 0,\; uv^ixy^iz \in L\). We consider two cases. First, if $vxy$ does not contain the \#, then it lies entirely on one side. We can pump once to change the string on this side, therefore making it not equivalent to the other side. In this case the new string $uv^2xy^2z \notin L$. In the other case, assume $vxy$ does contain the \#. There are two subcases. First, if the \# is in $x$, then $u$ consists of only 1's and $y$ consists of only 0's. So we can pump once and the two sides are unequal, so $uv^2xy^2z \notin L$. If either $u$ or $y$ contain the \#, we can pump down to 0 and remove it, thereby getting a string with no \# that cannot be in the language. We have now covered all cases and subcases, and arrived at strings not in L using the pumping lemma. Hence, L is not context free.


\subsection*{Problem 10}
If a CFG has $b$ symbols show that if there exists a string in the language that has a derivation of size greater than $2^b$ that
the language is infinite. Assume that the CFG is in Chomsky normal form.

Proof: Each step in a derivation represents a split in the parse tree. Therefore the derivation with at least $2^b + 1$ steps
represents a parse tree with $2^b + 1$ steps. Therefore the height of this parse tree is $b+1$, since we are in Chomsky normal
form, where each node has at most branching factor 2. And by the pigeonhole principle
we have a repeated symbol on the path from the root start symbol to the furthest leaf node or terminal. Therefore like we did
in the pumping lemma, we can repeat that symbol as many times as we wish to generate a new string in the language. Therefore we
have a infinite language.
 


\end{document}
