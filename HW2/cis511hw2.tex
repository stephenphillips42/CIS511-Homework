\documentclass[english]{article}

\usepackage[latin9]{inputenc}
\usepackage[letterpaper]{geometry}
\geometry{verbose,tmargin=1in,bmargin=1in,lmargin=1in,rmargin=1in}
\usepackage{amsmath}
\usepackage{amssymb}
\usepackage{tikz}
\usetikzlibrary{automata,positioning}

\tikzstyle{dir}=[->, very thick]
\tikzstyle{circ}=[draw, circle, very thick]

\title{CIS 511 Homework 2}
\author{Stephen Phillips, Dagaen Golomb}
\date{\today }


\begin{document}
\maketitle
\subsection*{Problem 1}
\subsubsection*{Part a}
First we show that $A = \{ \mathbf{a}^m \mathbf{b}^n \mathbf{c}^n | m,n \ge 0 \}$ is context free. The grammar is as follows:
\begin{align*}
S \rightarrow & XY \\
X \rightarrow & \varepsilon \; | \; aX \\
Y \rightarrow & \varepsilon \; | \; bYc \\
\end{align*}
Clearly this generates only strings in the language, and every string in $A$ can be mapped to this by the number of generations of $X$ and $Y$ corresponding to $m$ and $n$ respectively. The language $B = \{ \mathbf{a}^n \mathbf{b}^n \mathbf{c}^m | m,n \ge 0 \}$ is almost identical:
\begin{align*}
S \rightarrow & XY \\
X \rightarrow & \varepsilon \; | \; aXb \\
Y \rightarrow & \varepsilon \; | \; cY \\
\end{align*}
With the proof that for this being the same as $A$. So these two languages are context free, however their intersection $C = A \cap B = \{ \mathbf{a}^n \mathbf{b}^n \mathbf{c}^n | n \ge 0 \}$ is not context free, as we showed in class.

\subsubsection*{Part b}
We show that the complement of a context free is not necessarily context free. Suppose not. Then, for any context free languages $A$ and $B$, $A \cup B$ would be context free, since context free languages are closed under union. Then $\bar{A \cup B} = \bar{A} \cap \bar{B}$ would also be in the language. But since we assume that context free languages are closed under complement $\bar{A}$ and $\bar{B}$ would be context free. But since $A$ and $B$ are arbitrary that means the intersection of any two context free languages is context free, a contradiction. Therefore context free languages are not closed under complement.

\section*{Problem 2}
The context free grammar $G$ of $L = \{ w\#x | w^\mathcal{R} \textrm{is a substring of }x, x \in \{0,1\}^* \}$ is:
\begin{align*}
S \rightarrow & aSXaX \; | \; bSXbX \; | \; \# \\
X \rightarrow & \varepsilon \; | \; aX \; | \; bX \\
\end{align*}
Now we need prove this is correct, or in other words $L(G) \subseteq L$ and $L \subseteq L(G)$.
\begin{itemize}
\item $L(G) \subseteq L$. This generates strings of the form
      $\sigma_1 \sigma_2 \ldots \sigma_n \# \{a,b\}^* \sigma_n \{a,b\}^* \sigma_{n-1} \ldots \{a,b\}^* \sigma_1 \{a,b\}^* = w\#x$.
      Each generation of $S$ creates some $\sigma_i$ on the left and right, with the right having the reverse order of the left.
      The variable $X$ can generate any string in $\{a,b\}^*$, which generates the strings between the characters on the left side.
      So by construction $w^\mathcal{R}$ is a substring of $x$ as every $\sigma_i$ in $w$ is in $x$ and in the same order. Therefore
      the language this grammar generates is a subset of $L$
\item $L \subseteq L(G)$. Again the grammar generates strings of the form 
      $\sigma_1 \sigma_2 \ldots \sigma_n \# \{a,b\}^* \sigma_n \{a,b\}^* \sigma_{n-1} \ldots \{a,b\}^* \sigma_1 \{a,b\}^* = w\#x$.
      By definition if $x$ is a substring of $y$ then there exists $z_1, z_2, \ldots, z_n, z_{n+1} \in \{a,b\}^*$ such that 
      $y = z_1 x_1 \ldots z_n x_n z_{n+1}$. So for every string $s = w\#x \in L$ map $w_1$ to $\sigma_n$, and in general $w_i$ to
      $\sigma_{n-i+1}$ and then the $z_i$ map to the $\{a,b\}^*$. 
\end{itemize}

% I am not sure how good this proof is, Dagaen. If you can make it better please do so!

\section*{Problem 3}
Show that adding the rule \(S \rightarrow SS\) fails to show that context free languages are closed under Kleene star.

This fails because it is possible that the original grammar did not have $S$ able to go to a start symbol. For instance
the language \(L = \{ a^n b \mid n > 0 \}\) can be generated by the following grammar:
\[ S \rightarrow \; aS \; | \; ab \]
Adding the rule $SS$ makes this:
\[ S \rightarrow \; aS \; | \; ab \; | \; SS \]
Which is the language \(L' = \{ x^m \mid x \in L, m > 0 \}\) where as \(L^* = L' \cup \{\varepsilon\}\). 

\section*{Problem 4}
Changing the grammar to new add new start symbol $S'$ and add to the production rules:
\[ S' \rightarrow \; SS' \; | \; \varepsilon \]

This makes an arbitrary number of strings in the language concatinated, including 0. 

\section*{Problem 5}
Find a context free grammar for the language \(L = \{ x | \textrm{\#a's is 2 times \#b's}\}\)
\begin{align*}
S \rightarrow & 
\end{align*}

\section*{Problem 6}

\begin{align*}
S \rightarrow & 
\end{align*}




\end{document}
