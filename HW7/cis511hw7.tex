\documentclass[english]{article}

\usepackage[latin9]{inputenc}
\usepackage[letterpaper]{geometry}
\geometry{verbose,tmargin=1in,bmargin=1in,lmargin=1in,rmargin=1in}
\usepackage{amsmath}
\usepackage{amssymb}
\usepackage{tikz}
\usepackage{algpseudocode}
\usepackage{booktabs}

\usepackage{verbatim}
\usepackage{tkz-berge}
\usetikzlibrary{automata,positioning,arrows,shapes,petri,topaths}

\tikzstyle{dir}=[->, very thick]
\tikzstyle{circ}=[draw, circle, very thick]

\title{CIS 511 Homework 7}
\author{Stephen Phillips, Dagaen Golomb}
\date{April 15, 2015}


\begin{document}
\maketitle
\subsection*{Problem 1}
Show that $STRONGLY-CONNECTED = \{ \langle G \rangle \mid G \textrm{ is a graph that is strongly connected} \}$
is $NL$-complete

Since $NL =$ co-$NL$, and we know that $\overline{STRONGLY-CONNECTED}$ is in $NL$, then $STRONGLY-CONNECTED$ is in
co-$NL$, hence $NL$. 

\subsection*{Problem 2}
Show that $2SAT$ is $NL$-complete.

We show that $\overline{2SAT}$ is in co-$NL$, which means that since $NL =$ co-$NL$ that $2SAT$ is in $NL$. Given
an instance of $2SAT$ in CNF form, we can interpret the clauses as implications: $(a + b) \iff (\bar{a} \implies b)$.
We can `derive' one implication from another by using : $(u \implies v)(v \implies w) \iff (u \implies w)$.
Since there are only two variables per clause, if the formula is unsatisfiable, then we must be able to derive both
$(x \implies \bar{x})$ and $(\bar{x} \implies x)$, since if there was a chaine of implications that led to a 
contradiction then we could reduce the chain using derivations to get to these implications.
This in turn leads to:
\[ (x \implies \bar{x})(\bar{x} \implies x) 
\implies (\bar{x} + \bar{x})(x + x) 
\implies \bar{x}x \implies 0 
\]
We can do this by non-deterministically following the implications that derive $(x \implies \bar{x})$ then again
non-deterministically choose the ones that derive $(\bar{x} \implies x)$. If we find this line of implications, 
which we only need to store $O(\log n)$ bits for the intermediate implication we are using to derive the above.

We know that $\overline{PATH}$ is co-$NL$-complete, hence $NL$-complete. Therefore we want to show
$\overline{PATH} \le_L 2SAT$. To do this we consider given a graph $G$ we create the graph $G^\mathcal{R}$, the
graph with all the edges reversed. We know that there is a path from $s$ to $t$ in $G$ if and only if there is a
path from $t$ to $s$ in $G^\mathcal{R}$. We use this in the reduction. 

We do this by, given an instance of $\overline{PATH}$, meaning a graph $\langle G, s, t \rangle$, we create a boolean
formula $\phi$ from the edges. So $\forall (u,v) \in E$, we create clause $(u \implies v) = (\bar{u} + v)$ and clause
$(v' \implies u') = (\bar{v'} + u')$ (corresponding to $G^\mathcal{R}$, different variables), then and all these
clauses together. Except replace the variables for the nodes $s$ and $t$ that from the instance with literals $x$ and
$\bar{x}$ respectively, and also replace the variables $s'$ and $t'$ with $x$ and $\bar{x}$, respectively.
This can be done in log space since we only need to keep track of the two variables in the clause we are currently
reading, a constant amount of space. 

Why does this mapping work? 
If there is a path from $s$ to $t$, then we can derive $x \implies \bar{x}$ by following the implications on the path
from $s$ to $t$. Similarly on the reversed graph we can follow the path from $t$ to $s$ to derive
$\bar{x} \implies x$. Hence we can always derive both $x \implies \bar{x}$ and $\bar{x} \implies x$ when there is a
path from $s$ to $t$. If this is the case then we can derive our unsatisfiablility when we showed $2SAT$ was in $NL$.
Hence if there is a path, this formula is never satisfiable. If there is not a path from $s$ to $t$, the formula is
always satisfiable. We examine several cases:
\begin{enumerate}
\item $t$ has a path to $s$ but not the other way around. That means in the $G$ part of the formula we have a clause
  with $\bar{x} \implies x$, so we set $x$ to true, and all variables that $t$ can reach to true. For the variables
  corresponding to $G^\mathcal{R}$ we set to false since the implication $x \implies \bar{x}$ is in that set of
  clauses. Since we can vary the variables independently we can set both to true, and we can therefore set all the 
  clauses to true. 
\item $s$ and $t$ have no paths to each other at all. Then we are dealing with separate connected components, and 
  similarly to last case we can just set $x$ to a value and set the implications in the chain appropriately. 
\end{enumerate}
So this instance of $2SAT$ is satisfiable if and only if there is no path from $s$ to $t$ in $G$. 

\subsection*{Problem 3}
Give an example of an $NL$-complete context free language.

\subsection*{Problem 4}
Define $pad : \Sigma^* \times \mathcal{N} \rightarrow \Sigma^*\#^*$ as $pad(s,l) = s\#^l$. Define
the language $pad(A,f)$ for language $A$ and function $f : \mathcal{N} \rightarrow \mathcal{N}$ as
\[ pad(A,f) = \{ pad(s,f(|s|)) \mid s \in A \} \]
Show that if $A \in \textrm{TIME}(n^6)$ then $pad(A,n^2) \in \textrm{TIME}(n^3)$

Is this not just that padding extends the length of the string? Now if you run the TM for $A$ on the 
first part of the string, it will run in time $O(|s|^6)$, and since the length of our input is
$n = |s|^2 + |s| = O(|s|^2)$ this becomes $O(|s|^6) = O((|s|^2)^3) = O(n^3)$ (?)

\subsection*{Problem 5}
Prove using $pad$ from previous problem that if $NEXPTIME \neq EXPTIME$ then $P \neq NP$. 

\subsection*{Problem 6}
Show that for an $n$ variable polynomial $P$, with degree at most $d$, and total degree of $t$. We showed in class
that if you pick $r_1, \ldots, r_n$ uniformly and independently at random in a set $S$ then 
\[ \mathbf{Pr}[P(r_1,r_2,\ldots,r_n) = 0] \le \frac{nd}{|S|} \]
We now want to strengthen this result to:
\[ \mathbf{Pr}[P(r_1,r_2,\ldots,r_n) = 0] \le \frac{t}{|S|} \]


\subsection*{Problem 7}
Show if $NP \subseteq BPP$, then $NP = RP$

\subsection*{Problem 8}
Define a $ZPP$-machine as a probabilistic Turing Machine that can output 3 things: accept, reject, and ?.
A $ZPP$-machine $M$ decided a langauge $A$ if for every $x \in L$ it accepts with probability at least $2/3$, and
rejects with probability 0, for $x \notin L$ it rejects with probability $2/3$ and accepts with probability 0, and it
outputs $?$ on any input with probability at most $1/3$.


\end{document}

