\documentclass[english]{article}

\usepackage[latin9]{inputenc}
\usepackage[letterpaper]{geometry}
\geometry{verbose,tmargin=1in,bmargin=1in,lmargin=1in,rmargin=1in}
\usepackage{amsmath}
\usepackage{amssymb}
\usepackage{tikz}
\usepackage{algpseudocode}
\usepackage{booktabs}

\usepackage{verbatim}
\usepackage{tkz-berge}
\usetikzlibrary{automata,positioning,arrows,shapes,petri,topaths}

\tikzstyle{dir}=[->, very thick]
\tikzstyle{circ}=[draw, circle, very thick]

\title{CIS 511 Homework 6}
\author{Stephen Phillips, Dagaen Golomb}
\date{April 15, 2015}


\begin{document}
\maketitle
\subsection*{Problem 1}
Let $TQBF_{CNF} = \{ \phi \mid \phi \textrm{ is a TQBF with the part after the quantifiers being in CNF} \}$. We show
that $TQBF_{CNF}$ is PSPACE-Complete by reducing $TQBF$ to it in polynomial time.

% Didn't we show the equivalence of arbitrary boolean formulas and CNF boolean formulas in class? How to convert one
% To the other in polynomial time? Look here: http://en.wikipedia.org/wiki/Tseitin_transformation

So we just need to show that any boolean formula can be converted to an equivalent CNF formula in polynomial time.
We showed something similar to this in class in our reduction from SAT to 3SAT. Here we show this again.

Consider the formula like logical a gate network (unable to show diagram), with the operators acting as two input
gates. We can attach intermediate variables to the wires of the gates to change the formula into CNF form. We
can consider the network to only have AND, OR and NOT gates without loss of generality. We now find the formulas
that convert the gates into an equivalent CNF form, with existance qualifiers:
\begin{itemize}
\item AND gate:
\begin{align*}
xy \;\iff&\; \exists z : (z \implies (xy))(\bar{z} \implies \overline{(xy)}) & \textrm{Equivalent definitions} \\
\iff &\; \exists z : (\bar{z} + (xy))(z + \bar{x} + \bar{y}) & \textrm{DeMorgan's law and Definition of Implication}\\
\iff &\; \exists z : (\bar{z} + x)(\bar{z} + y)(z + \bar{x} + \bar{y}) & \textrm{Distributive law} \\
\end{align*}
\item OR gate:
\begin{align*}
x + y \;\iff&\;\exists z:(z \implies(x + y))(\bar{z}\implies \overline{(x + y)})&\textrm{Equivalent definitions} \\
\iff &\; \exists z : (\bar{z} + x + y)(z + \bar{x}\bar{y}) & \textrm{DeMorgan's law and Definition of Implication} \\
\iff &\; \exists z : (\bar{z} + x + y)(z + \bar{x})(z + \bar{y}) & \textrm{Distributive law} \\
\end{align*}
\item NOT gate:
\begin{align*}
\bar{x} \;\iff&\;\exists z:(z \implies x)(\bar{z}\implies \bar{x}) & \textrm{Equivalent definitions} \\
\iff &\; \exists z : (\bar{z} + x)(z + \bar{x}) & \textrm{DeMorgan's law and Definition of Implication} \\
\end{align*}
\end{itemize}
Thus for each gate we add in at most 3 more variables. If there are $n$ variables, and $m$ gates, this means the new
formula would be of size $O(m + n)$ still. You can also think of this in terms of the original formula instead of the
gates, There is a correspondence between gates and the $+$ or $\cdot$ operators, so it would still be the same factor
of three blowup.
% OK Not sure of this... I know it is still polynomial though.

\subsection*{Problem 2}
We are looking at the game of cat and mouse on a graph. The game is given an undirected graph $G$ and nodes $c$ and
$m$, the starting nodes for the cat and mouse respectively, and a special `hole' node $h$, the cat wants to get to
the same position as the mouse, and the mouse wants to get to the whole before that happens. The language is:

\begin{align*}
HAPPY-CAT =&\; \{ \langle G,c,m,h \rangle \mid \\
          &\;\;\; G,c,m,h \textrm{ form a game of cat and mouse and Cat has a winning strategy} \\
          &\;\;\;    \textrm{ if it moves first} \}
\end{align*}

% Wait... isn't this just a slightly more complicated version of PATH? Meaning we just need to know:
% If there is a path for both the cat and the mouse to the hole, and the cat's shortest path to the hole has length
% less than or equal to the mouses, then the cat wins.
First we get out the obvious case: If the mouse is closer to the hole than the cat is, then the mouse always wins.
This holds even when the graph is disconnected if you define the distance between nodes in disconnected components
to be infinity. So now we consider the cases where the mouse is further away or equal distance from the hole:
\begin{itemize}
\item The trivial case where mouse and cat start on the same node or are only one off - the cat always wins.
\item The mouse and cat are on different connected components, and the mouse and hole are on different connected
components. Clearly the game is a draw as no one can further their goals. 
\item The connected component is a tree (this will lead to the more general next case). Since the cat is closer,
and there are no alternative paths for the mouse to take, the mouse is trapped and the cat always wins. 
\item This case is more complicated. Say you remove the hole node. If the connected component the mouse is in this
new Graph is a tree (i.e. no cycles), then the cat has a winning strategy. Since it can get to the hole node first
then we are more or less back in the previous case where we are chasing the mouse on a tree so it is trapped
\item The final case is where there are cycles in the connected component where the mouse is. This means that if the
cat chased the mouse, they would run in circles forever. Since they are both logically perfect, the cat would never
allow the mouse to get to the hole (if nothing else, it could just sit on the hole forever). So this case is a draw.
\end{itemize}

All these cases can be checked in polynomial time, using a breadth first search for instance. Since there are only
a small number of these cases, this whole thing can be checked in polynomial time. 


% Next we get out of the way the more tedious cases where the graph is disconnected.
% \begin{itemize}
% \item If the cat, mouse, and hole are all in different connected components, it is a stalemate, since there is
% nothing any of them can do to move to their objectives. (Or if the cat and hole are on the same connected component
% and the mouse on a different one)
% \item If the mouse and hole are on one connected component, and the cat on another, the mouse will always win no
% matter what the cat does
% \end{itemize}

\subsection*{Problem 3}
We are considering 
\[ MIN-FORMULA = \{ \phi \mid \phi \textrm{ has no equivalent formula smaller than it}\} \]
We want to show that if $P = NP$, then $MIN-FORMULA \in P$.

We will heavily rely on the assumption that $P = NP$. First it is easy to see that to check that two formulas are
not equivalent is in NP: $\overline{EQUIV} =
\{ \langle \phi, \psi \rangle \mid \phi \textrm{ is not equivalent to }\psi \}$. The verifier is the set of inputs
that makes the two functions differ. Since we assume $P = NP$, we know that its complement $EQUIV$ is in $P$ as well. 
That means there must be a deterministic Turing machine $D_{EQUIV}$ that decides $EQUIV$ in polynomial time.

It is also easy to see that with $D_{EQUIV}$ it is easy to decide
\[ \overline{MIN-FORMULA} = \{ \phi \mid \phi \textrm{ has an equivalent formula smaller than it}\} \]
the complement
of what we are looking for. The verifier is the formula that is equivalent and smaller. Note that this only works as
a verifier since we assume $D_{EQUIV}$ runs in polynomial time. And since has a verifier it is in $NP$ and by 
assumption in $P$. So there is a deterministic Turing Machine $M$ that decides $\overline{MIN-FORMULA}$, and hence
we can complement its output and get a deterministic Turing Machine that decides ${MIN-FORMULA}$

In summary since we assume that $P = NP$, we can find the deterministic Turing Machines of the complements of our
languages of interest and then complement the output.

\subsection*{Problem 4}
We use the definition of $MIN-FORMULA$ from above.
\subsubsection{Part a}
We want to show that $MIN-FORMULA \in PSPACE$
% Is this not just enumerate all formulas less than this and check that they are equivalent by going through all 2^n
% formulas? 
Here is the algorithm:

\begin{algorithmic}
\Function{$M$}{$\phi$}
  \ForAll{$\psi$ boolean formulas smaller than $\phi$}
    \ForAll{Inputs $x_1,\ldots,x_n \in \{0,1\}$}
      \If{$\phi(x_1,\ldots,x_n) \neq \psi(x_1,\ldots,x_n)$}
        \State Break the loop
      \EndIf
    \EndFor
    \If{All inputs were the same}
      \State Reject
    \EndIf
  \EndFor
  \State Accept
\EndFunction 
\end{algorithmic}

This can be shown to be correct fairly straightforwardly, since it iterates through all formulas smaller than
$\phi$, and ensures that none of them are equivalent by checking iterating through all possible inputs.

This takes exponential time, but it only needs to store the boolean formulas and the input values,
and a constant amount of space to keep track of the solution whether the loop accepted space which is $O(m+n)$,
hence polynomial. Since we only use polynomial space, we are in $PSPACE$.

\subsubsection{Part b}
We will show that the following argument is wrong: $MIN-FORMULA \in coNP$ since if $\phi \notin MIN-FORMULA$ then
$\phi$ has a smaller equivalent formula which a NTM can guess.
% This doesn't work because it takes exponential time to verify that two formulas are the same.
In the previous problem we showed that $MIN-FORMULA \in P$ if $P = NP$. We used the fact that $P = NP$ to construct
a polynomial time Turing Machine to check that two formulas are equivalent. However, we do not know such a Turing
Machine, so we don't know how to verify in polynomial time if two formulas are equivalent. So even if we were shown
a smaller equivalent formula, we wouldn't know how to prove it except by iterating through all possible inputs.
So this proof is incomplete unless it described a way to do that.

\subsection*{Problem 5}
Show that the language of properly nested parenthesis and brackets is in $L$ (e.g. $([()()]([]))$ )

First we make sure the brackets and parenthesis are both balanced with respect to each other.
This is easy: keep a count of the number of outstanding brackets (or parentheses). Increment on each
opening bracket and decrement on each closing one. If the count ever goes negative, reject. If we
reach the end of the string and the count is non-zero, reject. This count number takes up O(log($n$))
space in the worst case (when all characters are opening brackets, for example, the count will be
the length of the string which can be represented in log($n$) bits).

Given that the previous two checks succeeded, we must now check that they occur in the correct order,
i.e. that the situation given in the example does not occur. Note that we can work on this recursively:
we can check that the first bracket set is correct, and assume that the whole string is correct if the
string between them is also correct. Therefore, we then check the next inner set of brackets for
correctness, and so on.

To do this, we keep track of the current position of the bracket we are checking and its type.
This takes log($n$) space for the position and a single space to remember the bracket type.
Note that we only need to check open brackets, since we know the string is already balanced.
We keep a count as we did before (including current opening symbol), except now we don't need to
distinguish between brackets and parentheses. When we first reach 0, we check that the next symbol
is the correct closing symbol for the symbol we are checking (using the stored variable of the
current bracket type). We scan the string and do this for each opening symbol.

To see that this is correct, see that the only way for this to fail is a situation similar to the
given example ([)]. We know the brackets are balanced with respect to each symbol and that the total
count is balanced. Therefore, the only way a string that makes it this far can fail is if it is
matched with an incorrect closing symbol. In such a situation, when the count reaches zero we will
be faced with an incorrect bracket and can reject. We only accept if we do not reject for all
opening symbols. Given this and the fact that the string is balanced, we can assert the string is
correct.

\subsection*{Problem 6}
Show that $UCYCLE = \{ \langle G \rangle \mid G \textrm{ is an undirected graph with a simple cycle} \}$
is in $L$.

We can check each node to see if it is in a cycle individually, keeping track of which node we are
testing using log($|V|$) space. Let us number each edge, say by order they appear on our input.
We can store an edge using log($|E|$) space.

The procedure is as follows: Store the current vertex we are testing, as well as the edge we leave
when testing. At each node, take the edge with the next highest labeling than the one we come in on.
Note this can be done by looking at the input tape and remembering just the last traversed edge. If
we enter on the largest labeled edge, leave through the smallest one. This in essence performs DFS.
To see why, notice how at each vertex we iterate through increasing edge numbers, so if a higher edge
is the one we are "looking for" we will find it. What if the edge is a lower label? Then we will exhaust
the higher labels, return on the highest label, and take the lowest one. We will then start iterating
through the lower numbers, effectively cycling through all edges with wrap-around.

We accept if we return to the vertex in question on a different edge than the one we left on.

To see why this works, lets say a graph has a cycle with edges 3, 7, and 10. The idea is to start
at edge 3 moving towards the node with edge 7. At this next node, there may be edges 4, 5, and 6.
However, in this case it will traverse any such edges and come back. When it does, it will by
definition take the next largest edge. Eventually, we will take edge 7. This process then continues,
with edge 10 eventually being taken. Since edge 10 is different from 3, we would accept.
Note this could work with other starting edges, the above is one illustration for example. In fact,
any cyclic shift of the ordered edges would work.

Using this process, we iterate through all vertices and try each incident edge as the starting edge.
If one finds a cycle, we are done. We respond reject only if all attempts fail.

This only requires storing the evaluated vertex, the edge we left on, and the label of the last
traversed vertex. All of these can be stored in log-space each, so the problem only requires
logarithmic space.

\end{document}

