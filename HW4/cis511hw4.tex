\documentclass[english]{article}

\usepackage[latin9]{inputenc}
\usepackage[letterpaper]{geometry}
\geometry{verbose,tmargin=1in,bmargin=1in,lmargin=1in,rmargin=1in}
\usepackage{amsmath}
\usepackage{amssymb}
\usepackage{tikz}
\usepackage{algpseudocode}
\usepackage{booktabs}
\usetikzlibrary{automata,positioning}

\tikzstyle{dir}=[->, very thick]
\tikzstyle{circ}=[draw, circle, very thick]

\title{CIS 511 Homework 4}
\author{Stephen Phillips, Dagaen Golomb}
\date{February 23, 2015}


\begin{document}
\maketitle
\subsection*{Problem 1}
We want to show that finding if a Turing Machine has a \textit{useless state}
is Turing Decidable. We formulate this as a language:
\[ L = \{ \langle M, q \rangle \mid
   q \in Q(M), \;
   \forall s \in \Sigma^* M(q) \, M\textrm{ does not enter }q \} \]

Now we reduce this to $A_{TM}$ to show that it is undecidable. First as usual, 
we suppose toward contradiction that there was a decider of this language, which
we will denote $D$. We build a Turing Machine in the following manner:
\begin{algorithmic}
\Function{N}{$\langle M, x \rangle$}
\State Build the description for the following machine
	\Function{A}{$y$}
		\State Ignore input $y$
		\State Simulate $M(x)$, and output what it outputs
	\EndFunction
\State Simulate $D(\langle A, q_A \rangle)$, where $q_A$ is the accept state
		of $D$, and output what it outputs
\EndFunction 
\end{algorithmic}

If $M(x)$ halts and accepts, then we eventually reach the accept state of $A$,
by construction. Also by construction we never reach the accept state if $M(x)$
loops or if it rejects. Since $D$ is a decider, and building the machine
description takes finite time, $N$ always halts. Therefore $N$ accepts if and
only if $D$ accepts, which in turn accepts if and only if
$\langle M,x \rangle \in A_{TM}$. Therefore, $N$ decides $A_{TM}$, a
contradiction. 

\subsection*{Problem 2}
\subsection*{Problem 3}
We want to show that the intersection of two context free languages is
undecidable using $A_{TM}$. 

\subsection*{Problem 4}
\subsection*{Problem 5}
Show that the language
\[ ISO = \{ \langle G, H \rangle \} \mid
		\textrm{$G$ and $H$ are isomorphic graphs} \]
is in NP.

To do this we need to show a verifier $V$ for $ISO$. As one might expect, the
certificate for a member of the language $\langle G, H \rangle$ is the
isomorphism $\phi: V_G \rightarrow V_H, v \rightarrow \phi(v)$ between them.
The size of such an isomorphism would be about $2n$ where $n$ is the number of
nodes in $G$ and $H$, so it polynomial in the size of $G$ and $H$. 
We also need to check that the verifier $V$ using this runs in polynomial
time. A simple algorithm to use this is the following:

\begin{algorithmic}
\Function{V}{$\langle G, H \rangle$,$\phi$}
\State If the number of nodes or edges in $G$ differ from the number of nodes
		or edges in $H$, reject
\For{Nodes $v$ in $G$}
	\State If the number of edges $(v,u)$ in $G$ differ from the number of
			edges $(\phi(v),u')$ in $H$, reject
	\For{Edges $(v,u)$ in $G$}
		\State If $(\phi(v),\phi(u))$ is not in $H$, reject
	\EndFor
\EndFor
\State Accept
\EndFunction 
\end{algorithmic}

By assering that the size of the graphs are the same, that each node $v$ has
the same degree as $\phi(v)$ and maps to the corresponding vertices, we have
shown that the isomorphism is correct. If there is no isomorphism then at
least one of these tests will fail. Therefore $V(\langle G, H \rangle,y)$
will accept for some input $y$ if and only if there is an isomorphism between
$G$ and $H$

\subsection*{Problem 6}
\subsection*{Problem 7}



\end{document}




