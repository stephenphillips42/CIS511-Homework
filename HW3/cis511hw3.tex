\documentclass[english]{article}

\usepackage[latin9]{inputenc}
\usepackage[letterpaper]{geometry}
\geometry{verbose,tmargin=1in,bmargin=1in,lmargin=1in,rmargin=1in}
\usepackage{amsmath}
\usepackage{amssymb}
\usepackage{tikz}
\usepackage{algpseudocode}
\usepackage{booktabs}
\usetikzlibrary{automata,positioning}

\tikzstyle{dir}=[->, very thick]
\tikzstyle{circ}=[draw, circle, very thick]

\newcommand{\ol}[1]{ \overline{#1} }

\newcommand{\ul}[1]{ \underline{#1} }

\title{CIS 511 Homework 3}
\author{Stephen Phillips, Dagaen Golomb}
\date{February 23, 2015}


\begin{document}
\maketitle
\subsection*{Problem 1}
Show a Turing Machine that accepts the language
 $L = \{ x \in {a,b,c}^* \mid 
         \textrm{$x$ contains more $a$'s than $b$'s and $c$'s combined} \}$

We define the Turing Machine 
$M = \langle Q,q_0,\Sigma,\Gamma,\delta,q_A, q_R \rangle $ as follows. First
we define the alphabets:

\[ \Sigma = \{a,b,c\} \]
\[ \Gamma = \{a,b,c,\dot{a},\dot{b},\dot{c}, \sqcup \} \]

The dotted symbols' indicate marked symbols which we have already read and
accounted for. We will define the transition function, with the set $Q$ being
implicitly defined by the states that appear in the transition function. First
step for this to work is to shift everything left so we know where the end of the 
tape is, like we did in class. After this, we have a blank symbol ($\sqcup$) at each
end of the tape.

The following table describes what to do in each state given what is read from the
tape. We provide a short description of the state's purpose to ease in understanding
how the machine works. The starting state is $q_0$ and the final states are $q_A$
and $q_R$ which halt and accept or reject, respectively.

\begin{center}
	\begin{tabular}{ l | c | c | c | c | c | c | c | r }
		& a & b & c & $\dot{a}$ & $\dot{b}$ & $\dot{c}$ & $\sqcup$ & Description\\
		\hline
		$q_0$ & $q_0$,$a$,R & $q_1$,$\dot{b}$,L & $q_1$,$\dot{c}$,L & $q_0$,$\dot{a}$,R & $q_0$,$\dot{b}$,R & $q_0$,$\dot{c}$,R & $q_4$,$\sqcup$,L & Scanning for a $b$ or $c$, so far $a=b+c$. \\ \hline
		$q_1$ & $q_1$,$a$,L & $q_1$,$b$,L & $q_1$,$c$,L & $q_1$,$\dot{a}$,L & $q_1$,$\dot{b}$,L & $q_1$,$\dot{c}$,L & $q_2$,$\sqcup$,R & Marked a $b$ or a $c$, return to left of tape.\\ \hline
		$q_2$ & $q_3$,$\dot{a}$,L & $q_2$,$b$,R & $q_2$,$c$,R & $q_2$,$\dot{a}$,R & $q_2$,$\dot{b}$,R & $q_2$,$\dot{c}$,R & $q_R$,$\sqcup$,R & Scan for an $a$ to balance $b$ and $c$'s.\\ \hline
		$q_3$ & $q_3$,$a$,L & $q_3$,$b$,L & $q_3$,$c$,L & $q_3$,$\dot{a}$,L & $q_3$,$\dot{b}$,L & $q_3$,$\dot{c}$,L & $q_0$,$\sqcup$,R & Marked an $a$, return to left of tape.\\ \hline
		$q_4$ & $q_A$,$a$,L & $q_4$,$b$,L & $q_4$,$c$,L & $q_4$,$\dot{a}$,L & $q_4$,$\dot{b}$,L & $q_4$,$\dot{c}$,L & $q_R$,$\sqcup$,R & No more $b$ or $c$'s, scan left for $a$'s.\\ \hline
		$q_R$ &&&&&&&& Rejecting state \\ \hline
		$q_A$ &&&&&&&& Accepting state \\
		\hline
	\end{tabular}
\end{center}

\subsection*{Problem 2}
Show that a language is recognizable by a queuing automaton if and only if it
is recognizable by a Turing Machine.
% This is the basic idea of the proof, however, I am not sure that this is
% rigorous enough to work
\begin{itemize}
\item ($ \impliedby $) If we have a Turing recognizable language, by
  definition there must be a Turing Machine that recognizes it. Therefore if
  we can make a queuing automaton replicate the actions of the Turing Machine
  we have show that Turing recognizable languages are recognized by queuing
  automata.

  To do this, we simplify the start and say that the queuing automaton pushes
  a `start of tape' symbol followed by the entire string $x$ into its queue before
  it starts computation. Now
\end{itemize}

\subsection*{Problem 3}
Show that Turing recognizable languages are closed under Kleene Star.

To do this we consider the Turing recognizable language $L$, and the machine
that recognizes it $M$.
On a given input $x$, there are a large but finite number of ways to split the
string into substrings. For a given partition, we can test if all the strings
are in $L$ using the following subroutine:
\begin{algorithmic}
\Function{A}{$x_1,x_2,\ldots,x_n$}
\State Create list to store which runs on strings $x_i$ have halted
\For{$i = 1,\ldots,\infty$}
	\For{$j$ in incomplete list}
		\State Simulate the next step of $M(x_j)$
		\If{$M(x_j)$ has halted}
			\State If $M$ rejects $x_j$, reject
			\State If $M$ accepts $x_j$, remove $j$ from incomplete list
		\EndIf
	\EndFor
	\If{Incomplete list empty}
		\State Accept
	\EndIf
\EndFor
\EndFunction 
\end{algorithmic}

This halts and accepts if and only if all the $x_i \in L$. It is fairly apparent,
as it rejects if even one of the $x_i$ gets rejected by $M$ and accepts only after
$M$ has halted and accepted all of the $x_i$. If neither of these happens,
one of the $x_i$ must have made $M$ loop forever, in which case $A$ also loops
forever since it will keep waiting until $M$ halts before it halts.

Using this subroutine, we can get the following to accept $L^*$:
\begin{algorithmic}
\Function{N}{$x$}
\State Let $n = |x|$
\For{$i = 1,\ldots,\infty$}
	\For{$j = 1,\ldots,n$}
		\For{All divisions of $x$ into $i$ strings, $x_1,x_2,\ldots,x_j$}
			\State Run one step of $A(x_1,x_2,\ldots,x_j)$
			\State If $A(x_1,x_2,\ldots,x_i)$ halts and accepts, accept
		\EndFor
	\EndFor
\EndFor
\State Reject
\EndFunction 
\end{algorithmic}

This machine $N$ accepts if and only if $x \in L^*$. 
\begin{itemize}
\item ($ \impliedby $) If $x \in L$ then by definition there exist strings
	$x_1, \ldots, x_n \in L$ such that $x = x_1 \ldots x_n$. As $N$
	goes through all possible partitions of $x$, this partition will be
	found and therefore $A(x_1,\ldots x_n)$ will accept, and therefore
	$N$ will accept.
\item ($ \implies $) If $N$ accepts $x$, then $A$ must have accepted
	for some $i \ge 1$ and with some partition $x = x_1 \ldots x_i$.
	But $A(x_1, \ldots, x_i)$ accepts if and only if all the $x_j$ are
	in $L$, which means by definition $x$ must be in $L^*$. If it wasn't
	then no such partition could have been formed and none of the 
	$A(x_1,\ldots,x_i)$ calls would have accepted. So therefore $N$ only
	accepts strings in $L^*$
\end{itemize}

As $N$ accepts all of $L^*$ and only accepts strings in $L^*$ the language
of $N$ must be $L^*$. Since this was done for arbitrary Turing recognizable
language $L$ it is true for all Turing recognizable languages.

\subsection*{Problem 4}
We show that a language $L$ is decidable if and only if it is enumerable in
the standard string order.
\begin{itemize}
\item ($ \impliedby $) If $L$ has an enumerable in the standard string
order, then there exists an Turing machine, the enumerator $E$, that
does exactly that: output all the strings in the language in the standard
string order. Using that enumerator we can create a decider $D$:
\begin{algorithmic}
\Function{$D$}{x}
	\For{All strings $y$ output by enumerator $E$}
		\State If $y = x$, accept
		\State If $y$ comes after $x$ in the standard string order, reject
	\EndFor
\EndFunction
\end{algorithmic}

If $E$ truly outputs all the strings in the language in the standard string
order, then if $x$ is in the language, it will eventually output $x$. Then
$D$ will accept when it is output by $E$. If it is not in the language, then
since $E$ outputs all stings in the language in the standard string order,
when we see a string $y$ that comes after $x$ in that order, we know we will
never see $x$ in the language, so we will reject. So therefore the language
of $D$ is $L$. And it is a decider, since $x$ has some finite
index in the standard ordering of strings, one of these two outcomes will
be reached in finite time. Therefore we will halt in finite time on all
strings.

\item ($ \implies $) If $L$ is decidable, there must be some decider $D$
that decided the language. Therefore we can make the following enumerator
$E$:
\begin{algorithmic}
\Function{$E$}{z}
	\State Ignore input $z$
	\For{All strings $x$ in the standard string order}
		\State If $D(x)$ accepts, output $x$
	\EndFor
\EndFunction
\end{algorithmic}

As $D$ is a decider, for every string $x$ it will halt in finite time.
Since it only accepts strings in $L$, $E$ can only output strings in $L$.
Since we are enumerating stings in the standard order, decide once 
whether or not to output them, we therefore must output the strings in
$L$ in the standard sting order, and $E$ functions as we wanted it to.

\end{itemize}

\subsection*{Problem 5}
Show that the language 
$L = \{ \langle G \rangle \mid
	\textrm{$G$ is a CFG over $\{0,1\}$ and $L \cap 1^* \neq \varnothing$}\}$
is Turing decidable.

We use a similar method as finding if a CFG generates the empty language.
To assist us, we have two types of marks on the symbols, $\dot{V}$ and
$\ddot{V}$. The first is to mark that we have seen it and it can generate
a string in $1^*$. The second is to mark we have seen it but it cannot
generate a string in $1^*$.  

The idea is that we look through the production rules and find strings in
$1^*$. We start at production rules that have right hand sides with only 
terminals, specifically terminal strings with only $1$ in it. For every
production rule we find with only $1$ in it, we mark the corresponding
variable $V$ with the first kind of mark $\dot{V}$. If there are 
variables that can generate terminal strings but not ones in $1^*$,
then we mark them with the second kind of symbol $\ddot{V}$. We denote
the first group $S_1$ and the second group $S_2$. 

Now we look at production rules with right hand sides with only symbols
in $S_1 \cup S_2$. If one of these right hand sides has only symbols
in $S_1$ we place the corresponding variable from the left hand side
into $S_1$. The rest of the variables we put into $S_2$. We repeat this
until we have all variables in $S_1 \cup S_2$. If the starting symbol
$S \in S_1$ then we know that this grammar generates strings in $1^*$,
and we accept. Otherwise we reject.

\subsection*{Problem 6}
Show that $L = \{ \langle A \rangle \mid L(A) = \varnothing \}$ is 
co-Turing recognizable.

To show this is co-Turing recognizable, we must show that $\bar{L}$ is Turing
recognizable. In other words $\bar{L} = \{ \langle A \rangle \mid
L(A) \neq \varnothing \}$. This is clearly recognizable, since we can just
simulate $A$ on all strings by the usual dove-tailing method. This is described
in detail below, described by the machine $M$.

\begin{algorithmic}
\Function{$M$}{x}
	\For{$i = 1,\ldots,\infty$}
		\For{$x$ in the first $i$ strings in $\Sigma^*$}
			\State Run $A(x)$ for $i$ steps
			\State If $A(x)$ halts and accepts, accept
		\EndFor 
	\EndFor
\EndFunction
\end{algorithmic}

So this machine will run forever on machines $A$ which have the empty language
which is fine since this is a recognizer so it does not need to halt on 
strings not in the language. However if $A$ has even one string in its
language, it will accept, since $A$ must halt and accept that string in some
finite time and therefore $M$ will as well.

As $\bar{L}$ is recognizable, $L$ must be co-Turing recognizable. 

\subsection*{Problem 7}
Show that a language $C$ is Turing recognizable if and only if there exists a 
decidable language $D$ such that
$C = \{ x \mid \exists y : \langle x, y \rangle \in D\}$.

The idea is that $y$ is the number of steps it takes for a recognizer of $C$
takes to accept $x$. The complete proof is as follows:
\begin{itemize}
\item ($ \impliedby $) If $D$ is a decidable language, then there must exist
	a Turing machine $N_D$ that decided the language. We construct the
	recognizer of $C$ as follows:
\begin{algorithmic}
\Function{$M_C$}{x}
	\For{All $y$ in $\Sigma^*$ in Lexicographic order}
		\State If $N_D(\langle x, y \rangle)$ accepts, accept 
	\EndFor 
\EndFunction
\end{algorithmic}

This loops forever for $x$ not in $C$, but as $N_D$ is guaranteed to halt
on every iteration of the loop, if $x \in C$ then we will find the $y$
such that $\langle x,y \rangle \in D$ and halt and accept. This satisfies
all that we need the decider $M_C$ to do, so we show that if such a
language $D$ exists, then $C$ is Turing recognizable.

\item ($ \implies $) If $C$ is recognizable, then we can easily construct
	such a language. Since $C$ is recognizable then there exists a
	Turing Machine $N_C$ that recognizes the language. Then we construct
	the corresponding $M_D$:
\begin{algorithmic}
\Function{$M_D$}{x,y}
	\State Let $n = |y|$
	\State Run $N_C(x)$ for $n$ steps
	\State If $N_C(x)$ halted and accepted within $n$ steps, accept
	\State Otherwise Reject
\EndFunction
\end{algorithmic}

Since this only runs $N_C$ for $|y|$ steps, it will halt on every input. If
$x \in C$ then that means $N_C$ must halt and accept after some number of 
steps, say $m$. That means that $\forall y : |y| > m, M_D(x,y)$ accepts. 
Therefore, the condition that there exists a $y$ is satisfied.
\end{itemize}

\subsection*{Problem 8}


\end{document}




