\documentclass[english]{article}

\usepackage[latin9]{inputenc}
\usepackage[letterpaper]{geometry}
\geometry{verbose,tmargin=1in,bmargin=1in,lmargin=1in,rmargin=1in}
\usepackage{amsmath}
\usepackage{amssymb}
\usepackage{tikz}
\usepackage{algpseudocode}
\usetikzlibrary{automata,positioning}

\tikzstyle{dir}=[->, very thick]
\tikzstyle{circ}=[draw, circle, very thick]

\title{CIS 511 Homework 3}
\author{Stephen Phillips, Dagaen Golomb}
\date{\today }


\begin{document}
\maketitle
\subsection*{Problem 1}
Show a Turing Machine that accepts the language
 $L = \{ x \in {a,b,c}^* \mid 
         \textrm{$x$ contains more $a$'s than $b$'s and $c$'s combined} \}$



\subsection*{Problem 2}
Show that a language is recognizable by a queueing automaton if and only if it
is recognizable by a Turing Machine.
% This is the basic idea of the proof, however, I am not sure that this is
% rigorous enough to work
\begin{itemize}
\item ($ \impliedby $) If we have a Turing recognizable language, by
  definition there must be a Turing Machine that recognizes it. Therefore if
  we can make a queueing automaton replicate the actions of the Turing Machine
  we have show that Turing recognizable languages are recognized by queueing
  automata.

  To do this, we simplfy the start and say that the queueing automaton pushes
  a `start of tape' symbol followed by the entire string $x$ into its queue before
  it starts computation. Now
\end{itemize}

\subsection*{Problem 3}
Show that Turing recognizable languages are closed under Kleene Star.

To do this we consider the Turing recognizable language $L$, and the machine
that recognizes it $M$.
On a given input $x$, there are a large but finite number of ways to split the
string into substrings. For a given partition, we can test if all the strings
are in $L$ using the following subroutine:
\begin{algorithmic}
\Function{A}{($x_1,x_2,\ldots,x_n$)}
\For{$i = 1,\ldots,\infty$}
	\State \For{$j = 1,\ldots,n$}
		\State Simulate the next step of $M(x_j)$
	\EndFor
\EndFor
\EndFunction 
\end{algorithmic}

\subsection*{Problem 4}

\subsection*{Problem 5}

\subsection*{Problem 6}

\subsection*{Problem 7}

\subsection*{Problem 8}


\end{document}
